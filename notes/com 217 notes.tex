\documentclass{report}
 
\usepackage[margin=1in]{geometry}
\usepackage{amsfonts}
\usepackage{amsmath}
\usepackage{amssymb}
\usepackage{amsthm}
\usepackage{CJK}
\usepackage{enumitem}
\usepackage{epsf}
\usepackage{fleqn}
\usepackage{float}
\usepackage{graphicx}
\usepackage{latexsym}
\usepackage{systeme}

% \input{mypreamble.tex}
\input{mymacros.tex}
\input{myletterfont.tex}

\newcommand{\dfn}[1]{\textbf{Definition.}\ #1}

\title{COM 217 Notes}
\author{Josh Park}
\date{Summer 2024}

\begin{document}
\maketitle
\chapter{Introduction} % 1

\chapter{Presenting with Confidence} % 2

\chapter{Delivering with Skill} % 3

\chapter{Assessing the Speaking Situation} % 4

\chapter{Information Literacy} % 5
    \section{Introduction}
        \dfn{According to the American Library Association, \emph{information literacy} is the ability to locate, evaluate, and effectively use information to support your ideas}
    
    \section{What is Fake News (and Why Is It Harmful?)}
    \dfn{\emph{Fake news} is ``fabricated information that mimics news media content in form but not organizational process or intent.''}
        \subsection{Seven types of Mis/Disinformation}
        \begin{enumerate}
            \item Fabricated Content - 100\% false, designed to be deceiving
            \item Satire/Parody - Designed to be humorous, but may trick people as it is presented in a serious way
            \item Misleading Content - Inaccurate information presented in a way to manipulate the truth
            \item Imposter Content - Mimics trustworthy sources
            \item False Connection - Headline doesn't match content of article, like clickbait
            \item False Context - Real information is shared with false context
            \item Manipulated Content - Real information is manipulated to deceive
        \end{enumerate}

        \subsection{Detecting Fake News/Misinformation}
            Confirmation bias in individuals can make it difficult to detect fake news. Some methods that can help are:
            \begin{enumerate}
                \item question everything you read, even if it seems true
                \item validate the information (check credentials of writer/source)
                \item use websites and software that can analyze the information
                \item follow guidelines outlined in this chapter
            \end{enumerate}
        
    \section{Locating Supporting Material}
        Knowing when to use information to support your claims is an important part of information literacy. Audiences who are more skeptical will require more information from more trustworthy sources to be properly convinced of your claims. 
        \subsection{Online Resources}
            \subsubsection{Google Scholar}
                Google Scholar is a modified Google search engine that can access databases of academic articles across disciplines. It has articles, theses, books, court opinions, information from academic publishers, professional societies, online repositories, and universities. Google Scholar can be linked to university libraries to search databases provided by those institutions as well.

            \subsubsection{Organizational websites}
                Websites for organizations like the American Society for the Prevention of Cruelty to Animals will often host tons of information about not only their organization but their industry as a whole, and may also have recent news articles that are relevant to their cause.

            \subsubsection{Crowd Sourcing Platforms}
                Platforms like Reddit and StackExchange allow anyone to ask questions and answer them. The quality of information can vary per community, but users can typically vote on other people's posts and responses, which helps filter out lots of misinformation.

            \subsubsection{Blogs}
                Similarly to organizational websites, blogs can contain a wealth of information related to a certain topic or field, but are not put under very strict editorial pressures so the information may not be great quality.

            \subsubsection{Twitter and Other Social Media}
                Can be a good way to search what other people are talking about (by use of hashtags, etc) and to follow experts in certain fields, but sources are often not given and there is no editorial process that guarantees quality of information. 

            \subsubsection{Online Newspapers and Magazines}
                Most traditional newspapers have rigorous editorial processes, so combined with the quick  turnaround on publishing, they can be great sources for accurate, recent information relevant to current issues

            \subsubsection{News Aggregators}
                Good way to look at lots of news articles from a large variety of outlets, helpful for catching up on current events. Popular news aggregators include:
                \begin{enumerate}
                    \item Science News
                    \item The Morning news
                    \item PopUrls
                    \item Flipboard
                    \item Google News
                    \item AP News
                \end{enumerate}

            \subsubsection{Databases}
                General databases can be accessed at almost any library and provides great access to a variety of useful materials. Popular databases include:
                \begin{enumerate}
                    \item \textbf{Nexis Uni:} Provides full-text access to U.S. and international newspaper articles, information on 80 million companies, information on federal and state court cases, laws, regulations, European Union law, patents, tax law, and law review articles.
                    \item \textbf{Academic Search Complete:} Provides a large collection of full-text journals. It includes peer-reviewed full text for STEM research as well as for the social sciences and humanities.
                    \item \textbf{Academic Search Premier:} General academic index that indexes more than 8,200 magazines and scholarly journals from every academic discipline and provides some full-text access.
                    \item \textbf{NewsWires:} Provides near real-time access to top world-wide news from Associated Press, United Press International, PR Newswire, Xinhua, CNN Wire, and Business Wire.
                \end{enumerate}

        \subsection{Scholarly Peer Reviewed Articles}
            Academic journals go through a rigorous peer review process before being published. These articles are very trustworthy and contain a lot of great information directly from an expert in their field.

        \subsection{Books}
            Books are connected to publishers, who may have good or bad reputations and thus the trustworthiness of a book can often be connected to what company published it. Books also take a long time to write, edit, publish, and distribute, so information published may be obsolete by the time it reaches the reader.

        \subsection{Librarians}
            Librarians are experts on finding trustworthy information and can most likely help you find relevant materials to your topic more quickly than alone.

        \subsection{Interviews}
            Interviewing an expert in a field is a great way to quickly get quality information regarding a subject. An important thing to remember is to quote the expert accurately and avoid using their claims out of context.
        
    \section{Evaluating Supporting Material}
        \subsection{Authority and Supporting Material}
        \subsection{Subject Expertise}
        \subsection{Societal Position}
        \subsection{Special Experience}
        \subsection{Other Considerations for Choosing Quality Sources}
        \subsubsection{Consult Reliable Sources}
        \subsubsection{Consider Primary and Secondary Sources}
        \subsubsection{Verify Your Material from More Than One Source}
        \subsubsection{Include a Global Perspective}
        \subsubsection{Ensure the Currency of Your Information}
        \subsubsection{Be Skeptical of Information}

    \section{Types of Supporting Material}
        \subsection{Statistics}
            \subsubsection{Where to Find Statistics}
            \subsubsection{Using Statistics Effectively}
            \subsubsection{Why and How Was the Data Collected?}
            \subsubsection{Has the Data Been Interpreted Correctly?}
                \subsubsection{Correlation vs Causation}
            \subsubsection{Report the Margin of Error}
            \subsubsection{Report the Correct Central Tendency}
            \subsubsection{Contextualize Your Statistics}
            \subsubsection{Simplify Your Statistics}
            \subsubsection{Use Statistics Strategically}
        
            \subsection{Opinion Statements}
            \subsubsection{Types of Opinion}
            \subsubsection{Tips for Using Opinion Correctly}
            \begin{enumerate}
                \item Provide Context for the Opinion
                \item Make Sure the Person Is an Expert
                \item Quoting vs. Paraphrasing
            \end{enumerate}
        
        \subsection{Examples}
            \subsubsection{Brief Examples}
            \subsubsection{Extended Examples}
            \subsubsection{Tips for Using Examples Effectively}

        \subsection{Images}
            \subsubsection{Places for Finding Images or Graphics}
            \subsubsection{Tips for Using Images Effectively}
            \begin{enumerate}
                \item Make Sure the Content and Tone Are Appropriate
                \item Ensure Ethical Use of Images
                \item Ensure Images Are Authentic
            \end{enumerate}

    \section{Using Supporting Material Ethically}

    \section{Providing Appropriate Citations}
        \subsection{Oral Citations During the Presentation}
        \subsubsection{Examples of proper citations}
        \begin{enumerate}
            \item Spending on pet care has grown 60 percent from 1996 to 2012 according to a 2017 article appearing in the New York Times.
            \item Cassia Denton, personal-training and group-exercises director for Balance Gym in Washington, was recently interviewed in January 2017 and said, “You should look for something that energizes you in a fitness routine.”
            \item According to their website, the National Rifle Association was founded in 1871.
            \item In Gary Taubes' 2017 New York Times best seller, The Case Against Sugar, he claims that sugar is a “dietary trigger.”
            \item “Apollo in 1969. Shuttle in 1981. Nothing in 2011. Our space program would look awesome to anyone living backwards thru time,” says, noted astrophysicist, Neil deGrasse Tyson.
            \item In a 2019 study published in the journal Current Biology, they reported data that provides evidence for episodic memory in dogs.
        \end{enumerate}

\chapter{Presentation Preparation} % 6

\chapter{Narratives and Storytelling} % 7

\chapter{Informative Presentations} % 8

\chapter{Visual Communication} % 9

\chapter{The Persuasive Process} % 10

\chapter{Persauasive Speaking} % 11

\chapter{Virtual Presentations} % 12
    \section{Introduction}
    
    \section{Why Virtual Presentations Matter}

    \section{Types of Virtual Presentations}
        \subsection{Asynchronous Presentations}
        \subsection{Synchronous Presentations}

    \section{Special Considerations}
        \subsection{Technology}
        \subsection{Recording the Presentation}
        \subsection{Lighting}
        \subsection{Positioning the Camera}
        \subsection{Camera Stability}
        \subsection{Audio Quality}
        \subsection{Background}
        \subsection{Delivery Factors}
        \subsection{Time}
        \subsection{Using Visual Aids}
    \section{Chapter Summary}

\end{document}
